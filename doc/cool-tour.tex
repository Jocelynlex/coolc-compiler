\documentstyle[11pt]{article}
%
% Copyright (c) 1995-1996 by Alex Aiken.  All rights reserved.
% Permission is granted to modify and distribute this document for
% for non-commercial purposes, so long as this copyright notice is retained
% in all copies.
%
% Side margins:
% Actual margin is 1 in + this number
\oddsidemargin -0.25in
\evensidemargin -0.25in

% Text width:
\textwidth 6.9in

% Top margin:
% Actual margin is 1.5 in + this number
\topmargin -.3in

% Text height:
\textheight 8.7in

% generally useful macros for writing
\newcommand{\TexDir}{/home3/aiken/tex}

% Three macros for defining:
%       Unix elements: filenames and program (sans serif)
%       Cool elements: literal tokens (typewriter)
%       C elements: function and variable names (boldface)
%
\def\U#1{{\sf{}#1}}
\def\S#1{{\tt{}#1}} % NB: we often use \verb+...+ for this also
\def\C#1{{\bf{}#1}}

% These allow switching interline spacing; the change takes effect immediately:

\makeatletter
\newcommand{\singlespacing}{\let\CS=\@currsize\renewcommand{\baselinestretch}{1}\tiny\CS}
\newcommand{\oneandahalfspacing}{\let\CS=\@currsize\renewcommand{\baselinestretch}{1.25}\tiny\CS}
\newcommand{\doublespacing}{\let\CS=\@currsize\renewcommand{\baselinestretch}{1.5}\tiny\CS}
%setspacingto sets the interline spacing to the value of its argument
% e.g., \setspacingto{1.5} is the same as \doublespacing
\newcommand{\setspacingto}[1]{\let\CS=\@currsize\renewcommand{\baselinestretch}{#1}\tiny\CS}
\makeatother

% nonumber
\newcommand{\nn}{\nonumber}

% Tab for hand-formatting:
\newcommand{\tab}{\hspace*{2em}}

% Angle brackets:
\newcommand{\la}{\langle}
\newcommand{\ra}{\rangle}

% s.t.
\newcommand{\st}{\mbox{\ s.t.\ }}

% otherwise
\newcommand{\ow}{\m{\rm otherwise}}

% if
\newcommand{\mif}{\m{\rm if\ }}

% Harpoons
\newcommand{\rh}{\rightharpoonup}
\newcommand{\lh}{\leftharpoonup}

% Denotational-semantics-style brackets and bottom:
\newcommand{\lbk}{\lbrack\!\lbrack}
\newcommand{\rbk}{\rbrack\!\rbrack}
\newcommand{\bottom}{\perp}

% Projection operator
\newcommand{\proj}{\!\downarrow\!}

% macros for mbox combined with another style
% (useful for changing typefaces in math mode)
\newcommand {\mboxbf}[1]{\mbox{{\bf #1}}}
\newcommand {\mboxit}[1]{\mbox{{\it #1}}}
\newcommand {\mboxem}[1]{\mbox{{\em #1}}}
\newcommand {\m}{\mbox}
\newcommand {\ch}{\rm}
	
% macro for creating a binary operator
%
% example:  \makebinop{\makebinop{\mybmod}{mod}
%	(duplicates the \bmod macro)
%
\def\makebinop#1#2{\def#1{\mskip-\medmuskip \mskip5mu
\mathbin{\rm #2} \penalty900 \mskip5mu \mskip-\medmuskip}}

% Environments for theorems, lemmas, etc.
\newtheorem{theorem}{Theorem}[section]
\newtheorem{lemma}[theorem]{Lemma}
\newtheorem{corollary}[theorem]{Corollary}
\newtheorem{definition}[theorem]{Definition}
\newtheorem{observation}[theorem]{Observation}
\newtheorem{fact}{Fact}[theorem]
\newtheorem{proposition}[theorem]{Proposition}
\newtheorem{example}[theorem]{Example}
\newtheorem{constraint}[theorem]{Constraint}
\newtheorem{axiom}[theorem]{Axiom}
\newtheorem{law}[theorem]{Law}
\newtheorem{algorithm}[theorem]{Algorithm}
\newtheorem{invariant}[theorem]{Invariant}

% Definition of the proof-environment:
\newenvironment{proof}{{\bf Proof:}\quad}{$\Box$}


%
% An environment for formatting programs. 
%
%	written by Hal Perkins
%	adapted by Charles Elkan	9/3/86
%	keyword macros by Anne Neirynck
%
% Usage :
%
% \begin{program}
% program text\\
% program text
% \end{program}
%
% The program environment is a tabbing environment with ten tab stops spaced
% evenly from the left of the page.  Initially the left margin is the second
% tab stop.  Use \+ to indent following lines one more tab stop, \- to undo
% the effect of \+, and \> at the beginning of a line to indent an extra tab.

\newlength{\pgmtab}          %  \pgmtab is the width of each tab in the
\setlength{\pgmtab}{2em}     %  program environment

% boxed program is like program, only boxed!
% This is useful for centering programs and preventing page breaks in programs.
% argument t or b is required.

\newenvironment{boxed-program}[1]{\begin{minipage}[#1]{9in}
\begin{tabbing}\hspace{\pgmtab}\=\hspace{\pgmtab}\=%
\hspace{\pgmtab}\=\hspace{\pgmtab}\=\hspace{\pgmtab}\=\hspace{\pgmtab}\=%
\hspace{\pgmtab}\=\hspace{\pgmtab}\=\hspace{\pgmtab}\=\hspace{\pgmtab}\=%
\hspace{\pgmtab}\=%
\kill}{\end{tabbing}\end{minipage}}

\newenvironment{program}{\begin{tabbing}\hspace{\pgmtab}\=\hspace{\pgmtab}\=%
\hspace{\pgmtab}\=\hspace{\pgmtab}\=\hspace{\pgmtab}\=\hspace{\pgmtab}\=%
\hspace{\pgmtab}\=\hspace{\pgmtab}\=\hspace{\pgmtab}\=\hspace{\pgmtab}\=%
\hspace{\pgmtab}\=%
\+\+\kill}{\end{tabbing}}

% The following commands should be used OUTSIDE math mode

\newcommand {\FUNCTION}{{\bf function\ }}
\newcommand {\BEGIN}{{\bf begin\ }}
\newcommand {\END}{{\bf end}}
\newcommand {\CASE}{{\bf case}}
\newcommand {\OF}{{\bf of}}
\newcommand {\SELECT}{{\bf select\ }}
\newcommand {\WHERE}{{\bf where\ }}
\newcommand {\DECLARE}{{\bf declare\ }}
\newcommand {\ARRAY}{{\bf array\ }}
\newcommand {\LET}{{\bf\ let\ }}
\newcommand {\IN}{{\bf\ in\ }}
\newcommand {\IF}{{\bf if\ }}
\newcommand {\THEN}{{\bf then\ }}
\newcommand {\ELSE}{{\bf else\ }}
\newcommand {\SKIP}{{\bf skip\ }}
\newcommand {\DO}{{\bf do\ }}
% OLD---don't use \OD
\newcommand {\OD}{{\bf od\ }}
\newcommand {\BY}{{\bf by\ }}
\newcommand {\LOOP}{{\bf loop\ }}
\newcommand {\WHILE}{{\bf while\ }}
\newcommand {\TO}{{\bf to\ }}
\newcommand {\DOWNTO}{{\bf down to\ }}
\newcommand {\FOR}{{\bf for\ }}
\newcommand {\FOREACH}{{\bf for each\ }}
\newcommand {\RETURN}{{\bf return\ }}
\newcommand {\REPEAT}{{\bf repeat\ }}
\newcommand {\UNTIL}{{\bf until\ }}
\newcommand {\LCOM}{$(\ast \;$}
\newcommand {\RCOM}{$\ast)$}
\newcommand {\GOTO}{{\bf goto\ }}

% The following commands are for use INSIDE math mode

\newcommand {\OP}[1]{\mbox{\sc #1}}
\newcommand {\op}[1]{\mbox{\sc #1}}
\newcommand {\mm}[1]{\mbox{\rm #1}\;}
\newcommand {\id}[1]{\mbox{\it #1\ }}

\newcommand {\ASSIGN}{\leftarrow }
\newcommand {\MIN}{\OP{min} }
\newcommand {\MOD}{\; {\bf{\rm mod}} \; } 
\newcommand {\LAMBDA}{{\bf \lambda\ }}
\newcommand {\FALSE}{{\em FALSE\ }}

% Miscellaneous notation

\newcommand {\And}{\wedge}
\newcommand {\Or}{\vee}

\newcommand {\thus}{{\dot{. \: .}\;}}

\newcommand {\bigO}[1]{{\cal O}(#1)}

\newcommand {\app}{\!\!:\!}
\newcommand {\hastype}{::}
\newcommand{\hast}{:}
\newcommand{\qt}[1]{\mbox{``#1''}}
\newcommand{\TexComment}[1]{}

\newcommand{\dq}{\m{\tt "}}
\newcommand{\flatqt}[1]{\m{\dq #1 \dq}}
\newcommand{\seq}{\subseteq}
\newcommand{\derives}{\vdash}

% for writing grammars
\newcommand{\grammar}{::=}
\newcommand{\gor}{\,|\,}

% macros for writing inference rules
\newcommand{\infrule}[2]{\displaystyle{\displaystyle\strut{#1}} \over %
                                        {\displaystyle\strut {#2}}}
\newcommand{\cinfrule}[3]{\parbox{14cm}{\hfil$\infrule{#1}{#2}$\hfil}\parbox{4cm}{$\,#3$\hfil}}

\begin{document}

\title{A Tour of the Cool Support Code\thanks{
Copyright \copyright 1995-1996 by Alex Aiken.
All rights reserved.}}  
\date{ }
\author{ }
\maketitle

\section{Introduction}

The Cool compiler project provides a number of basic data types to
make the task of writing a Cool compiler tractable in the timespan of
the course.  This document provides an overview of the Cool compiler
support code, which includes:

\begin{itemize}

\item a linked list data type;

\item a string table package;

\item a symbol table package;

\item miscellaneous utilities for the parser and lexer; 

\item a package for managing Cool abstract syntax trees;

\item routines for printing abstract syntax trees;

\item miscellaneous routines for handling multiple input files and
	command-line flags;

\item the runtime system.


\end{itemize}

This document should be read in conjunction with the source code.  With
the exception of the abstract syntax tree package (which is automatically
generated), there are also extensive comments in the source. 

The purpose of the Cool support code is 
to be easy to understand and use.  There are much more efficient
implementations of most of the data structures used in the system.  It
is recommended that students implementing a Cool compiler also stick
to simple, obviously correct, and perhaps inefficient
implementations---writing a compiler that works 
correctly is usually difficult enough.

The Cool system is written in C++ and it is assumed that the reader
has some familiarity with that language.  The base system deliberately
uses a simple subset of the language.  The heavily used features are:
classes, single inheritance, virtual functions, and templates.
Overloading is used very sparingly; with one exception, operator
overloading is avoided completely (the exception is the operator {\tt
<<}).  Destructors are not used and in fact memory management is
ignored completely.  Memory management is a very important part of
software development as it is currently practiced in industry, but
memory management is also tricky, error-prone, and very time consuming
to get completely right.  It is suggested that students also not
worry about memory management in writing their own Cool compilers.

\section{Lists}

The file {\tt list.h} implements a simple linked list datatype.  The operations
are similar to those provided in Lisp-family languages: a constructor {\tt List}
adds a new element to the front of a list, {\tt hd} returns the first element
of a list, and {\tt tl} returns the tail of a list.  Functions are also
provided for computing the length of a list, applying a function to every
element of a list, and printing a list.
Example uses of lists can be found 
in the implementation of both string tables and symbol tables.

\section{String Tables}

All compilers manage large numbers of strings such as
program identifiers, numerical constants, and string constants.
Often, many of these strings are the same. For example,
each identifier typically occurs many times in a program.
To ensure that string constants are stored compactly and 
manipulated efficiently, a specialized data structure, the
{\em string table}, is employed.

A string table is a lookup table that maintains a single copy of each
string.  The Cool string table class provides methods for inserting
and querying string tables in a variety of ways (see the file {\tt
stringtab.h}).  While production compilers use hashed data structures
to implement string tables, the Cool string tables are implemented
as lists (see {\tt stringtab\_functions.h}).  The components
of Cool string tables are of type {\tt Entry}.  Each {\tt Entry}
stores a string, the length of the string, and an integer index unique
to the string.

An important point about the structure of the Cool compiler is that
there are actually three distinct string tables: one for string
constants ({\tt stringtable}), one for integer constants ({\tt
inttable}), and one for identifiers ({\tt idtable}).  The code
generator must distinguish integer constants and string constants
from each other and from identifiers, because special code is produced
for each string constant and each integer constant in the program.
Having three distinct string tables makes this distinction easy.  Note
that each of the three tables has a different element type ({\tt
StrEntry}, {\tt IntEntry}, and {\tt IdEntry}), each of which is a
derived class of {\tt Entry}.  Throughout the rest of the compiler
(except parts of the code generator), a pointer to an {\tt Entry} is
called a {\tt Symbol}, irrespective of whether the symbol represents
an integer, string, or identifier.

Because string tables store only one copy of each string, comparing
whether two {\tt IntEntry}s, {\tt StrEntry}s, or {\tt IdEntry}s $\tt x$
and $\tt y$ represent the same string can be done simply by comparing
the two pointers {\tt x == y}.  Note that it does not make sense to
compare entries from different string tables (e.g., {\tt IntEntry}s
with {\tt StrEntry}s) as these are guaranteed to be different even if
the strings are the same.


Three
methods are provided to add elements to a table: \C{add\_string(char *s,int m)},
which adds a string \C{s} of at most \C{m} characters; \C{add\_string(char *s)},
which adds a string \C{s} to the table; and \C{add\_int(int i)}, which converts
integer \C{i} to a string and adds the string to the table.
Each of these methods returns a type derived from \C{Entry}
to describe the symbol table entry,  on which the method \C{get\_string}
is defined for extracting the entry's string.  Before using these functions you
should read the documentation in \U{stringtab.cc}.  If you don't use
the interface with the string table manager correctly, your program may
crash.

\section{Symbol Tables}

In addition to strings, compilers must also determine and manage the
scope of program names.  A symbol table is a data structure for
managing scope.  Conceptually, a symbol table is just another lookup
table.  The key is the symbol (the name) and the result is whatever
information has been associated with that symbol (e.g., the symbol's
type).

In addition to adding and removing symbols, symbol tables also support
operations for entering and exiting scopes and for checking whether an
identifier is already defined in the current scope.  The lookup
operation must also observe the scoping rules of the language; if
there are multiple definitions of identifier {\tt x}, the scoping
rules determine which definition a lookup of {\tt x} returns.  In most
languages, including Cool, inner definitions hide outer definitions.
Thus, a lookup on {\tt x} returns the definition of {\tt x} from the
innermost scope with a definition of {\tt x}.

Cool symbol tables are implemented as lists of scopes, where each
scope is a list of $\rm \la identifier,data \ra$ pairs.  The ``data''
is whatever data the programmer wishes to associate with each
identifier.  The symbol table operations are very straightforward to
define on this structure and are documented in {\tt symtab.h}.  An
example illustrating the use of symbol tables is in the file {\tt
symtab\_example.cc}.

\section{Utilities}

The files {\tt utilities.h} and {\tt utilities.cc} define a few
functions useful in writing and debugging a Cool parser and lexical
analyzer.  See the source code for documentation.

\section{Abstract Syntax Trees}

After lexical analysis and parsing, a Cool program is represented
internally by the Cool compiler as an abstract syntax tree.  The
project comes with a definition of Cool abstract syntax trees (ASTs)
built in.  The AST package is by far the largest piece of code in the
base system and requires the most time to learn.  The learning process
is made more complex because the AST code is generated automatically
from a specification in the file {\tt cool-tree.aps}.  While the
generated code is quite simple and regular in structure, it is also
devoid of comments.  This section serves as the documentation for the
AST package.


\subsection{Phyla and Constructors}

The AST data type provides, for each kind of Cool construct, a class
for representing expressions of that kind.  There is a class for {\tt
let} expressions, another class of {\tt +} expressions, and so on.
Objects of these classes are nodes in Cool abstract syntax trees.
For example, an expression $\tt e_1 + e_2$ is represented by a {\tt +}
expression object, which has two subtrees: one for the tree representing 
the expression $\tt e_1$ and one for the tree representing the
expression $\tt e_2$.

The Cool abstract syntax is specified in a language called APS.
In APS terminology, the various kinds of abstract syntax tree nodes
({\tt let}, {\tt +}, etc.) are called {\em constructors}.  (Don't
confuse this use of the term ``constructor'' with C++ constructors;
while similar, this is a slightly different meaning taken from
functional languages that predates C++.)  The form of the AST
is described by a set of {\it phyla}.  Each phylum has one or more
constructors.

Phyla are really just types.  That is, instead of having one large
group of undifferentiated constructors, the constructors are grouped
together according to function, so that, for example, the constructors
for expression ASTs are distinguished from the
constructors for class ASTs.  The phyla are defined at the
beginning of {\tt cool-tree.aps}:
\begin{verbatim}
module COOL begin
  phylum Program;

  phylum Class_;
  phylum Classes = LIST[Class_];

  phylum Feature;
  phylum Features = LIST[Feature];

  phylum Formal;
  phylum Formals = LIST[Formal];

  phylum Expression;
  phylum Expressions = LIST[Expression];

  phylum Case;
  phylum Cases = LIST[Case];

\end{verbatim}

From the definition it can be seen that there are two distinct kinds of
phyla: ``normal'' phyla and list phyla.  ``Normal'' phyla each have 
associated constructors; list phyla have a fixed set of list operations.

Each constructor takes typed arguments and returns a typed result.  The types
may either be phyla or any ordinary C++ type.  In fact, the phyla declarations
are themselves compiled into C++ class declarations by an APS compiler.
A sample constructor definition is
\begin{verbatim}
  constructor class_(name : Symbol; parent: Symbol; features : Features;
                     filename : Symbol) : Class_;
\end{verbatim}
This declaration specifies that the {\tt class\_\ } constructor\footnote{The name
 {\tt class\_} is chosen to avoid a conflict with the C++ keyword {\tt class}.} takes four
arguments: a {\tt Symbol} (a type identifier) for the class name, a
{\tt Symbol} (another type identifier) for the parent class, a
{\tt Features}, and a {\tt Symbol} for the filename in which the class definition
occurs.  The phylum {\tt Features} is defined to be a list of {\tt Feature}'s
by the declaration
\begin{verbatim}
   phylum Features = LIST[Feature];
\end{verbatim}
See Section~\ref{sec-lists} for a description of the operations defined on
AST lists.

The {\tt class\_} constructor returns an AST of type
(or phylum) {\tt Class\_}.  In {\tt cool.y} there is the following example of a
use of the {\tt class\_} constructor:
\begin{verbatim}
class   :  CLASS TYPEID INHERITS TYPEID IS optional_feature_list END ';'
           { $$ = class_($2,$4,$6,stringtable.add_string(curr_filename)); }
\end{verbatim}
The {\tt class\_} constructor builds a {\tt Class\_}  tree node with the
four arguments as children.  Because the phyla (types) of the arguments
are declared, the C++ type checker enforces that the {\tt class\_} constructor is 
applied only to arguments of the appropriate type.
See Section~\ref{sec-con} and {\tt cool-tree.aps}  to learn the definitions of the other constructors.\footnote{Comments in {\tt cool-tree.aps} begin with two hyphens ``-- --''.}

There is a real danger of getting confused because the same names are
used repeatedly for different entities in different contexts.  In the
example just above, small variations of the name {\tt class} are used
for a terminal ({\tt CLASS}), a non-terminal ({\tt class}), a constructor
({\tt class\_}), and a phylum ({\tt Class\_}).  These
uses are all distinct and mean different things.  There is
also a {\tt class\_} member of the {\tt union} declaration in
{\tt cool.y}, which means yet something else.  Most uses are
distinguished consistently by capitalization, but a few are not.
When reading the code it is important to keep in mind the role of each
symbol.


\subsection{AST Lists}
\label{sec-lists}

List phyla have a distinct set of operations for constructing and
accessing lists.  Note that the AST package uses its own definition of
lists, which is distinct from and has different operations than
the list type defined in {\tt list.h}.

For each phylum named {\em X}
there is a phylum called {\em X}s (except for {\tt Classes}, which is
a list of {\tt Class\_} nodes) of type {\tt List[X]}.  List functions are defined
automatically for each list.  For the {\tt Class\_} phylum some of the
list functions and methods are:
\begin{verbatim}
Classes  nil_Classes();
Classes  single_Classes(Class_);
Classes  append_Classes(Classes,Classes);
Class_   nth(int index);
int      len();
\end{verbatim}
These functions will be familiar to anyone with a passing knowledge of
Lisp or Scheme.  The function {\tt nil\_{\em phylum}s()} returns an
empty list of type {\em phylum}.  The function {\tt single\_{\em
phylum}s} makes a list of length 1 out of its {\em phylum}
argument. The function {\tt append\_{\em phylum}s} appends two lists of
{\em phylum}s.  The method {\tt nth} selects the {\tt index}'th element of
its list argument.  The method {\tt len} returns the length of the list.

AST lists also have a simple list iterator.  There is a method {\tt first}
that returns the index of the first element of the list, a predicate {\tt more}
that is false if its index argument is the last element of the list, and
a method {\tt next} that returns the next index of the list.  This iterator
is quite naive and inefficient; to find the $n$th element of the list,
up to $n$ elements of the list must be examined.  However, it is simple
to understand and use.
The list functions are defined in {\tt cool-X-tree.h} and {\tt tree.h}.
A typical use of the iterator functions to walk through a list $\tt l$ is:
\begin{verbatim}
for(int i = l->first(); l->more(i); i = l->next(i))
     { . . . do something with l->nth(i) . . . }
\end{verbatim}

\subsection{The AST Class Hierarchy}

With the exception of lists, all AST classes are derived from the
class {\tt tree\_node}.  All of the lists are lists of {\tt tree\_node}s.
The {\tt tree\_node} class and the AST list template are defined in 
{\tt tree.h}.

The {\tt tree\_node} class definition contains everything needed in an
abstract syntax tree node except information specific to particular
constructors.  There is a protected data member {\tt
line\_number}, the line number where the expression corresponding to
the AST node appeared in the source file.
 The line number is used by a Cool compiler to give good
error messages.  

Several functions are defined on all {\tt tree\_node}s.  The important functions
are: {\tt dump}, which pretty prints an AST and {\tt get\_line\_number},
which is a selector for the corresponding data member.

Each of the phyla is a class derived directly from {\tt tree\_node}.  As stated
previously, the phyla exist primarily to group related constructors together
and as such do not add much new functionality.

Each of the constructors is a class derived from the appropriate
phyla.  Each of the constructor classes defines a function of the same
name that can be used to build AST nodes.  The {\tt dump} function is
also defined automatically for each constructor.

\subsection{Class Members}
\label{sec-sel}

Each class definition of the tree package comes with a number of
members.  Some of the member functions are discussed above.
This section describes the data members and some more (but not all) of
the rest of the functions, as well as how to add new members
to the classes.  

Each constructor has data members defined for
each component of that constructor. The name of the member is the name of
the field in the constructor, and it is only visible to member functions
of the constructor's class or derived classes.
 For example, the {\tt class\_} constructor
has four data members:
\begin{verbatim}
Symbol name;
Symbol parent;
Features features;
Symbol filename;
\end{verbatim}
Here is a complete use of one member:
\begin{verbatim}
 Class_ c;
 Symbol p;

 Symbol Class_::get_parent() { return parent; }

 c = class(idtable.add_string("Foo",3),idtable.add_string("Bar"),nil_Features(),
           stringtable.add_string("filename"));
 p = c->get_parent();  // Sets p to the symbol for "Bar"
\end{verbatim}

It will be useful in writing a Cool compiler to extend the AST with
new functions such as {\tt get\_parent}. Simply modify the {\tt
cool-tree.h} file to add functions to the class of the appropriate
phylum or constructor.



\subsection{The Constructors}
\label{sec-con}

This section briefly describes each
constructor and its role in the compiler.
Each constructor corresponds to a portion of the Cool grammar.  The
order of arguments to a constructor follows the order in which symbols
appear in productions in the Cool syntax specification in the manual.
This correspondence between constructors and program syntax should
make clear how to use the arguments of constructors.  It
may be helpful to read this section in conjunction with {\tt cool-tree.aps}.

\begin{itemize}
\item {\tt program} \\
This constructor is applied at the end of parsing to the final list of classes.
The only needed use of this constructor is already  in the skeleton {\tt cool.y}.

\item {\tt class\_} \\
This constructor builds a class node from two types and a list of features.
See the examples above.

\item {\tt method} \\
This is one of the two constructors in the {\tt Feature} phylum.  Use
this constructor to build AST nodes for methods.  Note that the second
argument is a list of {\tt Formal}s.

\item {\tt attr} \\
This is the constructor for attributes.  The {\tt init} field is for the expression
that is the optional initialization.

\item {\tt formal} \\
This is the constructor for formal parameters in method definitions.  The field names are
self-explanatory.

\item {\tt branch} \\
This is the single constructor in the {\tt Case} phylum.  A branch of a {\tt case}
expression has the form
\begin{verbatim}
name : typeid => expr;
\end{verbatim}
which corresponds to the field names in the obvious way.  Use this constructor
to build an AST for each branch of a {\tt case} expression.

\item {\tt assign} \\
This is the constructor for assignment expressions.

\item {\tt static\_dispatch} and {\tt dispatch} \\
There are two different kinds of dispatch in Cool and they have distinct constructors.
See the CoolAid for a discussion of static vs. normal dispatch.  Note
there is a shorthand for dispatch that omits the {\tt self} parameter. Don't
use the {\tt no\_expr} constructor in place of {\tt self}; you need to fill in
the symbol for {\tt self} for the rest of the compiler to work correctly.

\item {\tt cond} \\
This is the constructor for {\tt if-then-else} expressions.

\item {\tt loop}\\
This is the constructor for {\tt loop-pool} expressions.

\item {\tt typcase}\\
This constructor builds an AST for a {\tt case} expression.  Note that the
second argument is a list of case branches (see the {\tt branch} constructor
above).

\item{\tt block} \\
This is the constructor for {\tt begin-end} expressions.

\item {\tt let}\\
This is the constructor for {\tt let} expressions. Note that the
{\tt let} constructor only allows one identifier.  When parsing a {\tt let}
expression with multiple identifiers, it should be transformed  into
nested {\tt let}s with single identifiers, as described in the semantics
for {\tt let} in the CoolAid.

\item {\tt plus}\\
This is the constructor for $\tt +$ expressions.

\item {\tt sub}\\
This is the constructor for $\tt -$ expressions.

\item {\tt mul}\\
This is the constructor for $\tt \ast$ expressions.


\item {\tt divide}\\
This is the constructor for $\tt /$ expressions.

\item {\tt neg}\\
This is the constructor for $\tt \;\tilde{ }\;$ expressions.

\item {\tt lt}\\
This is the constructor for $\tt <$ expressions.

\item {\tt eq}\\
This is the constructor for $\tt =$ expressions.

\item {\tt leq}\\
This is the constructor for $\tt <=$ expressions.

\item {\tt comp}\\
This is the constructor for $\tt not$ expressions.

\item {\tt int\_const}\\
This is the constructor for integer constants.


\item {\tt bool\_const}\\
This is the constructor for boolean constants.

\item {\tt string\_const}\\
This is the constructor for string constants.

\item {\tt new\_}\\
This is the constructor for {\tt new} expressions.

\item {\tt isvoid}\\
This is the constructor for {\tt isvoid} expressions.

\item {\tt no\_expr}\\
This constructor takes no arguments.  Use {\tt no\_expr}
where constructor arguments are missing because an optional expression
is omitted, except for a missing {\tt self} in a dispatch expression
(see the discussion of dispatch above).

\item {\tt object} \\
This constructor is for expressions that are just object identifiers.
Note that object identifiers are used in many places in the syntax,
but there is only one production for object identifiers as expressions.

\end{itemize}

\subsection{Tips on Using the Tree Package}

There are a few common errors people make using a tree package.
\begin{itemize}

\item The tree package implements an abstract data type.  Violating
 the interface (e.g., by casting, pointer arithmetic, etc.)
invites disaster.  Stick to the interface as it is defined.
When adding new members to the class declarations, be careful that
those members do not perturb the interface for the existing functions.

\item The value {\tt NULL} is not a valid component of any AST.
Never use {\tt NULL} as an argument to a constructor.
Use {\tt nil\_{\em phylum}\(\)} to create empty lists.

\item All tree nodes and lists are distinct.  A test such as
\begin{verbatim}
if (x == nil_Expression()) { ... }
\end{verbatim}
is always false, because {\tt nil\_Expression()} creates a new empty list
each time it is used.  To check whether a list is empty, use the
{\tt len} method (see {\tt tree.h}).

\item It is also pointless to compare with childless tree nodes.
For example,
\begin{verbatim}
if (x == no_expr()) { ... }
\end{verbatim}
is always false, because {\tt no\_expr} creates a new AST each time
it is called.  Define a virtual method to determine
the constructor of {\tt x} (see Section~\ref{sec-sel}).

\item The tree package functions perform checks to ensure that trees
are used correctly.  If something bad is detected, the function
{\tt fatal\_error} is invoked to terminate execution.  To determine
where the problem lies it is best to use the debugger {\tt dbx} or {\tt gdb}.  Set a breakpoint on {\tt fatal\_error} and use the
{\tt where} command to find out what routine caused the error.
\end{itemize}

\section{The Runtime System}
\label{sec-runtime}

The runtime system consists of a set of hand-coded assembly language
functions that are used as subroutines by Cool programs.  Under {\tt
spim} the runtime system is in the file {\tt trap.handler}.  The use
of the runtime system is a concern only for code generation, which
must adhere to the interface provided by the runtime system.

The runtime system contains four classes of routines: 
\begin{enumerate}
\item startup code, which invokes the main method of the
main program;

\item the code for methods of predefined classes
(\C{Object}, \C{IO}, \C{String});

\item a few special procedures needed by Cool programs to test objects
	for equality and handle runtime errors;

\item the garbage collector.
\end{enumerate}

The Cool runtime system is in the file \U{\~{}cs164/lib/trap.handler};
it is loaded automatically whenever \U{spim}/\U{xspim} is invoked. 
Comments in the file explain how the predefined functions are called.

The following sections describe what the Cool runtime system assumes about
the generated code, and what the runtime system provides to the generated
code. Read Section 13 of the \U{CoolAid} manual for a formal description
of the execution semantics of Cool programs.

\subsection{Object Header}
The first three 32-bit words of each object are assumed to contain a
class tag, the object size, and a pointer for dispatch information.
In addition, the garbage collector requires that 
the word immediately before an object contain -1; this word is not part of the
object.

\begin{figure}
\begin{center}
\begin{tabular}{|l|l|}
\hline
offset -4 & Garbage Collector Tag\\
\hline
offset 0 & Class tag\\
\hline
offset 4 & Object size (in 32-bit words)\\
\hline
offset 8 & Dispatch pointer\\
\hline
offset 12$\ldots$ & Attributes \\
\hline
\end{tabular}
\end{center}
\caption{Object layout.}
\label{fig0}
\end{figure}

Figure~\ref{fig0} shows the layout of a Cool object; the offsets are
given in numbers of bytes.  The garbage collection tag is -1.  The
class tag is a 32-bit integer identifying the class of the object. The
runtime system uses the class tag in equality comparisons between
objects of the basic classes and in the abort functions to index a
table containing the name of each class.

The object size field and garbage collector tag are maintained by the
runtime system; only the runtime system should create new objects.
However, {\em prototype objects} (see below) must be coded directly by
the code generator in the static data area, so the code generator
should initialize the object size field and garbage collector tag of
prototypes properly.  Any statically generated objects must also
initialize these fields.

The dispatch pointer is never actually used by the runtime system.
Thus, the structure of dispatch information is not fixed.  You should
design the structure and use of the dispatch information for your code
generator. In particular, the dispatch information should be used to
invoke the correct method implementation on dynamic dispatches.

For \C{Int} objects, the only attribute is the 32-bit value of the
integer. For \U{Bool} objects, the only attribute is the 32-bit value
1 or 0, representing either true or false.  The first attribute of
\U{String} objects is an object pointer to an \U{Int} object
representing the size of the string. The actual sequence of ASCII
characters of the string starts at the second attribute (offset 16),
terminates with a 0, and is then padded with 0's to a word boundary.

The value {\em void} is a null pointer and is represented by the
32-bit value 0.  All uninitialized variables (except variables of type
\U{Int}, \U{Bool}, and \U{String}; see the {\em CoolAid}) are set to
{\em void} by default.

\subsection{Prototype Objects}

The only way to allocate a new object in the heap is to use the {\tt
Object.copy} method.  Thus, there must be an object of every class
that can be copied.  For each class $X$ in the Cool program, the code
generator should produce a skeleton $X$ object in the data area; this
object is the prototype of class $X$.

For each prototype object the garbage collection tag, class tag,
object size, and dispatch information must be set correctly.  For the
basic classes {\tt Int}, {\tt Bool}, and {\tt String}, the attributes
should be set to the defaults specified in the {\em CoolAid}.  For the other
classes the attributes of the prototypes may be whatever you find
convenient for your implementation.

\subsection{Stack and Register Conventions}

The primitive methods in the runtime system expect arguments in register
\C{\$a0} and on the stack. Usually \C{\$a0} contains the \C{self} object
of the dispatch. Additional arguments should be on top of the stack, first
argument pushed first (an issue only for \C{String.substr}, which takes
two arguments). Some of the primitive runtime procedures expect
arguments in particular registers.

Figure~\ref{fig1} shows which registers are used by the runtime system. The
runtime system may modify any of the scratch registers without restoring them
(unless otherwise specified for a particular routine). The heap
pointer is used to keep track of the next free word on the heap, and
the limit pointer is used to keep track of where the heap ends. These
two registers should not be modified or used by the generated
code---they are maintained entirely in the runtime system. All other registers,
apart from \C{\$at},\C{\$sp},\C{\$ra}, remain unmodified.

\begin{figure}
\begin{center}
\begin{tabular}{|l|l|}
\hline
Scratch registers & \C{\$v0},\C{\$v1},\C{\$a0}--\C{\$a2},\C{\$t0}--\C{\$t2}\\
\hline
Heap pointer   & \C{\$gp} \\
\hline
Limit pointer & \C{\$s7}\\
\hline
\end{tabular}
\end{center}
\caption{Usage of registers by the runtime system.}
\label{fig1}
\end{figure}
\subsection{Labels Expected}

The Cool runtime system refers to the fixed labels listed in
Figure~\ref{fig10}.  Each entry describes what the runtime system
expects to find at a particular label and where (code/data segment)
the label should appear.

\begin{figure}
\begin{center}
\begin{tabular}{|l|l|l|}
\hline
\C{Main\_protObj}	& The prototype object of class \C{Main}	& Data\\
\C{Main\_init}	& Code that initializes an object of class \C{Main} &
Code\\
		& passed in \C{\$a0} &\\
\C{Main.main}	& The main method for class \C{Main}	& Code\\
		& \C{\$a0} contains the initial \C{Main} object & \\
\C{Int\_protObj}	& the prototype object of class \C{Int}	& Data\\
\C{Int\_init}	& code that initializes an object of class \C{Int} &
Code\\
		& passed in \C{\$a0} 	&\\
\C{String\_protObj}	& the prototype object of class \C{String}	& Data\\
\C{String\_init}	& code initializing an object of class \C{String} &
Code\\
		& passed in \C{\$a0} 	&\\
\C{\_int\_tag}	& a single word containing the class tag for the \C{Int}
class & Data\\
\C{\_bool\_tag}	& a single word containing the class tag for the \C{Bool}
class & Data\\
\C{\_string\_tag}	& a single word containing the class tag for the \C{String}
class & Data\\
\C{class\_nameTab}	& a table, which at index (class tag) $\ast$ 4 contains a pointer  & Data \\
		&  to a \C{String} object containing the name of the class
associated &\\
		&  with the class tag &\\
\C{bool\_const0}	& the \C{Bool} object representing the boolean value false & Data\\
\hline
\end{tabular}
\end{center}
\caption{Fixed labels.}
\label{fig10}
\end{figure}

There is no need for code that initializes an object of class \C{Bool}
if the generated code contains definitions of both \C{Bool} objects
in the static data area.
The easiest way to ensure these labels are correct in the generated code 
is to adopt the following naming conventions:
\begin{center}
\begin{tabular}{ll}
\verb#<class>_init#	&	for init code of class
\verb#<class>#\\
\verb#<class>.<method>#	&	for method \verb#<method># code of class
\verb#<class>#\\
\verb#<class>_protObj#	&	for the prototype object of class
\verb#<class>#\\
\end{tabular}
\end{center}
Finally, Figure~\ref{fig2} lists labels defined in the runtime system that are of
interest to the generated code.


\begin{figure}
\begin{center}
\begin{tabular}{|l|l|}
\hline
\C{Object.copy}	& A procedure returning a fresh copy of the object
passed in \C{\$a0} \\
		& Result will be in \C{\$a0}\\
\C{Object.abort}	& A procedure that prints out the class name of the
object in \C{\$a0} \\
		& Terminates program execution\\
\C{Object.type\_name}	& Returns the name of the class of object
passed in \C{\$a0} as a string object\\
		& Uses the class tag and the table class\_nameTab\\
\hline
\C{IO.out\_string}	& The value of the string object on top of the stack
is printed to \\
		& the terminal. Does not modify \C{\$a0}.\\
\C{IO.out\_int}	& The integer value of the Int object on top of the
stack is printed\\
		&  to the terminal. Does not modify \C{\$a0}.\\
\C{IO.in\_string}	& Reads a string from the terminal and returns the read
string object\\
		& in \C{\$a0}.\\
		& (The newline that terminates the input is not part
of the string)\\
\C{IO.in\_int}	& Reads an integer from the terminal and returns the read
int object\\
		& in \C{\$a0}. \\
\hline
\C{String.length}	& Returns the integer object which is the length of the string
object\\
		& passed in \C{\$a0}. Result in \C{\$a0}.\\
\C{String.concat}	& Returns a new string, made from concatenating the
string object on top \\
		& of the stack to the string object in \C{\$a0}. Return
value in \C{\$a0}\\
\C{String.substr}	& Returns the substring of the string object passed in
\C{\$a0}, from index~i\\
		& with length~l. The length is defined by the integer
object on top of\\
		& the stack, and the index by the integer object on
the stack below~l.\\
		& Result in \C{\$a0}.\\
\hline
\C{equality\_test}	& Tests whether the objects passed in \$t1 and \$t2
have the same\\
		& primitive type \{Int,String,Bool\} and the same
value. If they do,\\
		& the value in \C{\$a0} is returned, otherwise \$a1 is
returned.\\
\C{\_dispatch\_abort}	& Called when a dispatch is attempted on a void object.
Prints the line \\
               & number, from  \C{\$t1}, and filename, from \C{\$a0}, at
which the dispatch \\
               & occurred, and aborts.\\
\C{\_case\_abort}	& Should be called when a case statement has no match. \\
		& The class name of the object in \C{\$a0} is printed, and
execution halts.\\
\C{\_case\_abort2}       & Called when a case is attempted on a void object.
Prints the line \\
               & number, from  \C{\$t1}, and filename, from \C{\$a0}, at
which the dispatch \\
               & occurred, and aborts.\\
\hline
\end{tabular}
\end{center}
\caption{Labels defined in the runtime system.}
\label{fig2}
\end{figure}

\subsection{Execution Startup}
On startup, the following things happen:
\begin{enumerate}
\item
   A fresh copy of the \C{Main} prototype object is made on the heap and
   then initialized by a call to \C{Main\_init}.

   The code generator must define \C{Main\_init}.  \C{Main\_init}
should execute all initialization code of \C{Main}'s parent classes
and finally execute the initializations of attributes in \C{Main} (if
there are any). 

\item
   Control is transferred to \C{Main.main}, passing a pointer to the newly
   created \C{Main} object in  register \C{\$a0}. Register \C{\$ra} contains the
   return address.

\item
   If control returns from \C{Main.main}, execution halts with the message
``COOL program successfully executed''.
\end{enumerate}

\section{The Garbage Collector}

The Cool runtime environment includes two different garbage
collectors, a generational garbage collector and a stop-and-copy
collector.  The generational collector is the one used for programming
assignments; the stop-and-copy collector is not currently used.  The
generational collector automatically scans memory for objects that may
still be in use by the program and copies them into a new (and
hopefully much smaller) area of memory.

Generated code must contain definitions specifying which of several
possible configurations of the garbage collector the runtime system should
use.  The location \C{\_MemMgr\_INITIALIZER} should contain a pointer
to a initialization routine for the garbage collector and 
\C{\_MemMgr\_COLLECTOR} should contain a pointer to code for a collector.
The options either no collection, generational collection, or stop-and-copy
collection; see comments in the \U{trap.handler} for the names of the
\C{INITIALIZER} and \C{COLLECTOR} routines for each case.  If the
location \C{\_MemMgr\_TEST} is non-zero and a garbage collector is enabled,
the garbage collector is called on every memory allocation, which is useful
for testing that garbage collection and code generation are working properly
together.

The collectors assumes every even value on the stack that is a valid heap
address is a pointer to an object.  Thus, a code generator must ensure
that even heap addresses on the stack are in fact pointers to objects.
Similarly, the collector assumes that any value in an object that is
a valid heap address is a pointer to an object (the exceptions are objects
of the basic classes, which are handled specially).

The collector updates registers automatically as part of a garbage collection.
Which registers are updated is determined by a register mask that can be
reset.  The mask should have bits set for whichever registers hold heap
addresses at the time a garbage collection is invoked; see the file
\U{trap.handler} for details.

Generated code must notify the collector of every assignment to an
attribute.  The function {\tt \_GenGC\_Assign} takes an updated
address $a$ in register {\tt \$a1} and records information about $a$
that the garbage collector needs.  If, for example, the attribute at
offset 12 from the {\tt \$self} register is updated, then a correct
code sequence is
\begin{verbatim}
sw      $x 12($self)
addiu   $a1 $self 12
jal     _GenGC_Assign
\end{verbatim}
Calling {\tt \_GenGC\_Assign} may cause a garbage collection.  Note
that if garbage collector is {\em not} being used it is equally important
{\em not} to call {\tt \_GenGC\_Assign}.  

\end{document}

