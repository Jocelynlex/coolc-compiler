\documentstyle[handout,11pt]{article}
%
% Copyright (c) 1995-1996 by Alex Aiken.  All rights reserved.
% Permission is granted to modify and distribute this document for
% for non-commercial purposes, so long as this copyright notice is retained
% in all copies.
%
% Side margins:
% Actual margin is 1 in + this number
\oddsidemargin -0.25in
\evensidemargin -0.25in

% Text width:
\textwidth 6.9in

% Top margin:
% Actual margin is 1.5 in + this number
\topmargin -.3in

% Text height:
\textheight 8.7in

% generally useful macros for writing
\newcommand{\TexDir}{/home3/aiken/tex}

% These allow switching interline spacing; the change takes effect immediately:

\makeatletter
\newcommand{\singlespacing}{\let\CS=\@currsize\renewcommand{\baselinestretch}{1}\tiny\CS}
\newcommand{\oneandahalfspacing}{\let\CS=\@currsize\renewcommand{\baselinestretch}{1.25}\tiny\CS}
\newcommand{\doublespacing}{\let\CS=\@currsize\renewcommand{\baselinestretch}{1.5}\tiny\CS}
%setspacingto sets the interline spacing to the value of its argument
% e.g., \setspacingto{1.5} is the same as \doublespacing
\newcommand{\setspacingto}[1]{\let\CS=\@currsize\renewcommand{\baselinestretch}{#1}\tiny\CS}
\makeatother

% nonumber
\newcommand{\nn}{\nonumber}

% Tab for hand-formatting:
\newcommand{\tab}{\hspace*{2em}}

% Angle brackets:
\newcommand{\la}{\langle}
\newcommand{\ra}{\rangle}

% s.t.
\newcommand{\st}{\mbox{\ s.t.\ }}

% otherwise
\newcommand{\ow}{\m{\rm otherwise}}

% if
\newcommand{\mif}{\m{\rm if\ }}

% Harpoons
\newcommand{\rh}{\rightharpoonup}
\newcommand{\lh}{\leftharpoonup}

% Denotational-semantics-style brackets and bottom:
\newcommand{\lbk}{\lbrack\!\lbrack}
\newcommand{\rbk}{\rbrack\!\rbrack}
\newcommand{\bottom}{\perp}

% Projection operator
\newcommand{\proj}{\!\downarrow\!}

% macros for mbox combined with another style
% (useful for changing typefaces in math mode)
\newcommand {\mboxbf}[1]{\mbox{{\bf #1}}}
\newcommand {\mboxit}[1]{\mbox{{\it #1}}}
\newcommand {\mboxem}[1]{\mbox{{\em #1}}}
\newcommand {\m}{\mbox}
\newcommand {\ch}{\rm}
	
% macro for creating a binary operator
%
% example:  \makebinop{\makebinop{\mybmod}{mod}
%	(duplicates the \bmod macro)
%
\def\makebinop#1#2{\def#1{\mskip-\medmuskip \mskip5mu
\mathbin{\rm #2} \penalty900 \mskip5mu \mskip-\medmuskip}}

% Environments for theorems, lemmas, etc.
\newtheorem{theorem}{Theorem}[section]
\newtheorem{lemma}[theorem]{Lemma}
\newtheorem{corollary}[theorem]{Corollary}
\newtheorem{definition}[theorem]{Definition}
\newtheorem{observation}[theorem]{Observation}
\newtheorem{fact}{Fact}[theorem]
\newtheorem{proposition}[theorem]{Proposition}
\newtheorem{example}[theorem]{Example}
\newtheorem{constraint}[theorem]{Constraint}
\newtheorem{axiom}[theorem]{Axiom}
\newtheorem{law}[theorem]{Law}
\newtheorem{algorithm}[theorem]{Algorithm}
\newtheorem{invariant}[theorem]{Invariant}

% Definition of the proof-environment:
\newenvironment{proof}{{\bf Proof:}\quad}{$\Box$}

% Definition of an eqnarray-like environment with an extra
% column for comments
\newcommand{\eeqnarray}[1]{\[\begin{array}{rcll} #1 \end{array}\]}

% Definition of a eqnarray-like environment for proofs of the
% form 
%     init     
%=>   step1   reason1
%=>   step2   reason2
%...
\newcommand{\peqnarray}[1]{\[\begin{array}{cll} #1 \end{array}\]}

%
% An environment for formatting programs. 
%
%	written by Hal Perkins
%	adapted by Charles Elkan	9/3/86
%	keyword macros by Anne Neirynck
%
% Usage :
%
% \begin{program}
% program text\\
% program text
% \end{program}
%
% The program environment is a tabbing environment with ten tab stops spaced
% evenly from the left of the page.  Initially the left margin is the second
% tab stop.  Use \+ to indent following lines one more tab stop, \- to undo
% the effect of \+, and \> at the beginning of a line to indent an extra tab.

\newlength{\pgmtab}          %  \pgmtab is the width of each tab in the
\setlength{\pgmtab}{2em}     %  program environment

% boxed program is like program, only boxed!
% This is useful for centering programs and preventing page breaks in programs.
% argument t or b is required.

\newenvironment{boxed-program}[1]{\begin{minipage}[#1]{9in}
\begin{tabbing}\hspace{\pgmtab}\=\hspace{\pgmtab}\=%
\hspace{\pgmtab}\=\hspace{\pgmtab}\=\hspace{\pgmtab}\=\hspace{\pgmtab}\=%
\hspace{\pgmtab}\=\hspace{\pgmtab}\=\hspace{\pgmtab}\=\hspace{\pgmtab}\=%
\hspace{\pgmtab}\=%
\kill}{\end{tabbing}\end{minipage}}

\newenvironment{program}{\begin{tabbing}\hspace{\pgmtab}\=\hspace{\pgmtab}\=%
\hspace{\pgmtab}\=\hspace{\pgmtab}\=\hspace{\pgmtab}\=\hspace{\pgmtab}\=%
\hspace{\pgmtab}\=\hspace{\pgmtab}\=\hspace{\pgmtab}\=\hspace{\pgmtab}\=%
\hspace{\pgmtab}\=%
\+\+\kill}{\end{tabbing}}

% The following commands should be used OUTSIDE math mode

\newcommand {\FUNCTION}{{\bf function\ }}
\newcommand {\BEGIN}{{\bf begin\ }}
\newcommand {\END}{{\bf end}}
\newcommand {\CASE}{{\bf case}}
\newcommand {\OF}{{\bf of}}
\newcommand {\SELECT}{{\bf select\ }}
\newcommand {\WHERE}{{\bf where\ }}
\newcommand {\DECLARE}{{\bf declare\ }}
\newcommand {\ARRAY}{{\bf array\ }}
\newcommand {\LET}{{\bf\ let\ }}
\newcommand {\IN}{{\bf\ in\ }}
\newcommand {\IF}{{\bf if\ }}
\newcommand {\THEN}{{\bf then\ }}
\newcommand {\ELSE}{{\bf else\ }}
\newcommand {\SKIP}{{\bf skip\ }}
\newcommand {\DO}{{\bf do\ }}
% OLD---don't use \OD
\newcommand {\OD}{{\bf od\ }}
\newcommand {\BY}{{\bf by\ }}
\newcommand {\LOOP}{{\bf loop\ }}
\newcommand {\WHILE}{{\bf while\ }}
\newcommand {\TO}{{\bf to\ }}
\newcommand {\DOWNTO}{{\bf down to\ }}
\newcommand {\FOR}{{\bf for\ }}
\newcommand {\FOREACH}{{\bf for each\ }}
\newcommand {\RETURN}{{\bf return\ }}
\newcommand {\REPEAT}{{\bf repeat\ }}
\newcommand {\UNTIL}{{\bf until\ }}
\newcommand {\LCOM}{$(\ast \;$}
\newcommand {\RCOM}{$\ast)$}
\newcommand {\GOTO}{{\bf goto\ }}

% The following commands are for use INSIDE math mode

\newcommand {\OP}[1]{\mbox{\sc #1}}
\newcommand {\op}[1]{\mbox{\sc #1}}
\newcommand {\mm}[1]{\mbox{\rm #1}\;}
\newcommand {\id}[1]{\mbox{\it #1\ }}

\newcommand {\ASSIGN}{\leftarrow }
\newcommand {\MIN}{\OP{min} }
\newcommand {\MOD}{\; {\bf{\rm mod}} \; } 
\newcommand {\LAMBDA}{{\bf \lambda\ }}
\newcommand {\FALSE}{{\em FALSE\ }}

% Miscellaneous notation

\newcommand {\And}{\wedge}
\newcommand {\Or}{\vee}

\newcommand {\thus}{{\dot{. \: .}\;}}

\newcommand {\bigO}[1]{{\cal O}(#1)}

\newcommand {\app}{\!\!:\!}
\newcommand {\hastype}{::}
\newcommand{\hast}{:}
\newcommand{\qt}[1]{\mbox{``#1''}}
\newcommand{\TexComment}[1]{}

\newcommand{\dq}{\m{\tt "}}
\newcommand{\flatqt}[1]{\m{\dq #1 \dq}}
\newcommand{\seq}{\subseteq}
\newcommand{\derives}{\vdash}

% for writing grammars
\newcommand{\grammar}{::=}
\newcommand{\gor}{\,|\,}

% macros for writing inference rules
\newcommand{\infrule}[2]{\displaystyle{\displaystyle\strut{#1}} \over %
                                        {\displaystyle\strut {#2}}}
\newcommand{\cinfrule}[3]{\parbox{14cm}{\hfil$\infrule{#1}{#2}$\hfil}\parbox{4cm}{$\,#3$\hfil}}


\begin{document}
\handout{3}{2}{Programming Assignment I \\ Due Tuesday, Sept. 6}

This assignment asks you to write a short Cool program.  The purpose
is to acquaint you with the Cool language and to give you experience with some of
the tools used in the course.  This assignment will {\em not} be
done with a partner; you should turn in your own individual work.  All
future programming assignments will be done in teams of two.  Remember
that there is a 25\% penalty for each day or part of a day that an
assignment is late.

A machine with only a single stack for storage is a {\em stack machine}.
Consider the following very primitive language for programming  
a stack machine: \\
\begin{center}
\begin{tabular}{r|l}
{\em Command} & {\em Meaning} \\ \hline
{\em int}  & push the integer {\em int}  on the stack \\
+ & push a `+' on the stack \\
s & push an `s' on the stack \\
e & evaluate the top of the stack (see below) \\
d & display contents of the stack \\
x & stop
\end{tabular}
\end{center}

The `d' command simply prints out the contents of the stack, one element
per line, beginning with the top of the stack.
The behavior of the `e' command depends on the contents of the stack
when `e' is issued:
\begin{itemize}
\item If `+' is on the top of the stack, then the `+'
	is popped off the stack, the following two integers are popped and added, 	 and the result is pushed back on the stack.

\item If `s' is on top of the stack, then the `s' is popped and
	the following two items are swapped on the stack.

\item If an integer is on top of the stack or the stack is empty, the
	stack is left unchanged.
\end{itemize}

The following examples show the effect of the `e' command in various
situations; the top of the stack is on the left:
\[
\begin{array}{lcl}
\m{\em stack before} & & \m{\em stack after} \\
+ \; 1 \; 2 \; 5 \; s \ldots & \;\; & 3 \; 5 \; s \ldots \\
s \; 1 \; + \; + \,\; 99 \ldots & & + \,\; 1\; + \; 99 \\
1 \; + \; 3 \ldots & & 1 \; + \; 3 \ldots \\
\end{array}
\]

You are to implement an interpreter for this language in Cool.  Input
to the program is a series of commands, one command per line.  Your
interpreter should prompt for commands with {\tt >}.  Your program
need not do any error checking: you may assume that all commands are
valid and that the appropriate number and type of arguments are on the
stack for evaluation. You may also assume that the input integers are
unsigned.

You are free to implement this program in any style you choose.
However, in preparation for building a Cool compiler, we recommend that
you try to develop an object-oriented solution.  One approach is to
define a class {\tt Stack} with a number of generic operations, and
then to define subclasses of {\tt Stack}, one for each kind of command
in the language.  These subclasses define operations specific to each
command, such as how to evaluate that command, display that command, etc.
If you wish, you may use the classes defined in {\tt
atoi.cl} in the {\tt \~{ }cs164/examples} directory to perform string
to integer conversion.

We wrote a solution in approximately 130 lines of Cool source code.
This information is provided to you as a rough measure of the amount
of work involved in the assignment---your solution may be either
substantially shorter or longer.

{\em Sample session.} The following is a sample compile and run of our solution.
\begin{verbatim}
%coolc stack.cl atoi.cl
%spim -file stack.s
SPIM Version 5.4 of Jan. 17, 1994
Copyright 1990-1994 by James R. Larus (larus@cs.wisc.edu).
All Rights Reserved.
See the file README a full copyright notice.
Loaded: /home/n/cs164/lib/trap.handler
>1
>+
>2
>s
>d
s
2
+
1
>e
>e
>d
3
>x
COOL program successfully executed
\end{verbatim}

{\em Getting and turning in the assignment.} 
Create a working directory and {\tt cd} into it.  From there, type
\begin{verbatim}
% make -f ~cs164/assignments/PA1/Makefile
\end{verbatim}
This command creates several files you will need in the directory.
Follow the directions in the README file.  The README file explains how to
turn in your assignment when you are finished; it also contains a few questions
you are to answer as part of the assignment.

{\em Extra Credit.} There is a chance that you will discover a bug in
our Cool compiler.  We will award extra credit for legitimate bug
reports; to get credit, send a bug report to {\tt mjacoby@cs}.  Your
report must include all of the needed Cool source and a transcript of
a terminal session showing how to reproduce the bug (use the {\tt
script} command).  There are a number of ways the compiler can
potentially fail: the compiler may dump core, the generated code may
be incorrect, the compiler may refuse to accept a legal program, it
may accept an illegal program, etc.  {\em Please be sure you have
found a bug before submitting a report!} The course staff are the
final arbiters of what is a ``bug'' and what is a ``feature''.  Credit
usually will be awarded only to the first person to report a bug.


\end{document}

