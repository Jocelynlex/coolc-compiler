\documentstyle[11pt,handout]{article}
%
% Copyright (c) 1995-1996 by Alex Aiken.  All rights reserved.
% Permission is granted to modify and distribute this document for
% for non-commercial purposes, so long as this copyright notice is retained
% in all copies.
%
% Side margins:
% Actual margin is 1 in + this number
\oddsidemargin -0.25in
\evensidemargin -0.25in

% Text width:
\textwidth 6.9in

% Top margin:
% Actual margin is 1.5 in + this number
\topmargin -.3in

% Text height:
\textheight 8.7in

%% generally useful macros for writing
\newcommand{\TexDir}{/home3/aiken/tex}

% These allow switching interline spacing; the change takes effect immediately:

\makeatletter
\newcommand{\singlespacing}{\let\CS=\@currsize\renewcommand{\baselinestretch}{1}\tiny\CS}
\newcommand{\oneandahalfspacing}{\let\CS=\@currsize\renewcommand{\baselinestretch}{1.25}\tiny\CS}
\newcommand{\doublespacing}{\let\CS=\@currsize\renewcommand{\baselinestretch}{1.5}\tiny\CS}
%setspacingto sets the interline spacing to the value of its argument
% e.g., \setspacingto{1.5} is the same as \doublespacing
\newcommand{\setspacingto}[1]{\let\CS=\@currsize\renewcommand{\baselinestretch}{#1}\tiny\CS}
\makeatother

% nonumber
\newcommand{\nn}{\nonumber}

% Tab for hand-formatting:
\newcommand{\tab}{\hspace*{2em}}

% Angle brackets:
\newcommand{\la}{\langle}
\newcommand{\ra}{\rangle}

% s.t.
\newcommand{\st}{\mbox{\ s.t.\ }}

% otherwise
\newcommand{\ow}{\m{\rm otherwise}}

% if
\newcommand{\mif}{\m{\rm if\ }}

% Harpoons
\newcommand{\rh}{\rightharpoonup}
\newcommand{\lh}{\leftharpoonup}

% Denotational-semantics-style brackets and bottom:
\newcommand{\lbk}{\lbrack\!\lbrack}
\newcommand{\rbk}{\rbrack\!\rbrack}
\newcommand{\bottom}{\perp}

% Projection operator
\newcommand{\proj}{\!\downarrow\!}

% macros for mbox combined with another style
% (useful for changing typefaces in math mode)
\newcommand {\mboxbf}[1]{\mbox{{\bf #1}}}
\newcommand {\mboxit}[1]{\mbox{{\it #1}}}
\newcommand {\mboxem}[1]{\mbox{{\em #1}}}
\newcommand {\m}{\mbox}
\newcommand {\ch}{\rm}
	
% macro for creating a binary operator
%
% example:  \makebinop{\makebinop{\mybmod}{mod}
%	(duplicates the \bmod macro)
%
\def\makebinop#1#2{\def#1{\mskip-\medmuskip \mskip5mu
\mathbin{\rm #2} \penalty900 \mskip5mu \mskip-\medmuskip}}

% Environments for theorems, lemmas, etc.
\newtheorem{theorem}{Theorem}[section]
\newtheorem{lemma}[theorem]{Lemma}
\newtheorem{corollary}[theorem]{Corollary}
\newtheorem{definition}[theorem]{Definition}
\newtheorem{observation}[theorem]{Observation}
\newtheorem{fact}{Fact}[theorem]
\newtheorem{proposition}[theorem]{Proposition}
\newtheorem{example}[theorem]{Example}
\newtheorem{constraint}[theorem]{Constraint}
\newtheorem{axiom}[theorem]{Axiom}
\newtheorem{law}[theorem]{Law}
\newtheorem{algorithm}[theorem]{Algorithm}
\newtheorem{invariant}[theorem]{Invariant}

% Definition of the proof-environment:
\newenvironment{proof}{{\bf Proof:}\quad}{$\Box$}

% Definition of an eqnarray-like environment with an extra
% column for comments
\newcommand{\eeqnarray}[1]{\[\begin{array}{rcll} #1 \end{array}\]}

% Definition of a eqnarray-like environment for proofs of the
% form 
%     init     
%=>   step1   reason1
%=>   step2   reason2
%...
\newcommand{\peqnarray}[1]{\[\begin{array}{cll} #1 \end{array}\]}

%
% An environment for formatting programs. 
%
%	written by Hal Perkins
%	adapted by Charles Elkan	9/3/86
%	keyword macros by Anne Neirynck
%
% Usage :
%
% \begin{program}
% program text\\
% program text
% \end{program}
%
% The program environment is a tabbing environment with ten tab stops spaced
% evenly from the left of the page.  Initially the left margin is the second
% tab stop.  Use \+ to indent following lines one more tab stop, \- to undo
% the effect of \+, and \> at the beginning of a line to indent an extra tab.

\newlength{\pgmtab}          %  \pgmtab is the width of each tab in the
\setlength{\pgmtab}{2em}     %  program environment

% boxed program is like program, only boxed!
% This is useful for centering programs and preventing page breaks in programs.
% argument t or b is required.

\newenvironment{boxed-program}[1]{\begin{minipage}[#1]{9in}
\begin{tabbing}\hspace{\pgmtab}\=\hspace{\pgmtab}\=%
\hspace{\pgmtab}\=\hspace{\pgmtab}\=\hspace{\pgmtab}\=\hspace{\pgmtab}\=%
\hspace{\pgmtab}\=\hspace{\pgmtab}\=\hspace{\pgmtab}\=\hspace{\pgmtab}\=%
\hspace{\pgmtab}\=%
\kill}{\end{tabbing}\end{minipage}}

\newenvironment{program}{\begin{tabbing}\hspace{\pgmtab}\=\hspace{\pgmtab}\=%
\hspace{\pgmtab}\=\hspace{\pgmtab}\=\hspace{\pgmtab}\=\hspace{\pgmtab}\=%
\hspace{\pgmtab}\=\hspace{\pgmtab}\=\hspace{\pgmtab}\=\hspace{\pgmtab}\=%
\hspace{\pgmtab}\=%
\+\+\kill}{\end{tabbing}}

% The following commands should be used OUTSIDE math mode

\newcommand {\FUNCTION}{{\bf function\ }}
\newcommand {\BEGIN}{{\bf begin\ }}
\newcommand {\END}{{\bf end}}
\newcommand {\CASE}{{\bf case}}
\newcommand {\OF}{{\bf of}}
\newcommand {\SELECT}{{\bf select\ }}
\newcommand {\WHERE}{{\bf where\ }}
\newcommand {\DECLARE}{{\bf declare\ }}
\newcommand {\ARRAY}{{\bf array\ }}
\newcommand {\LET}{{\bf\ let\ }}
\newcommand {\IN}{{\bf\ in\ }}
\newcommand {\IF}{{\bf if\ }}
\newcommand {\THEN}{{\bf then\ }}
\newcommand {\ELSE}{{\bf else\ }}
\newcommand {\SKIP}{{\bf skip\ }}
\newcommand {\DO}{{\bf do\ }}
% OLD---don't use \OD
\newcommand {\OD}{{\bf od\ }}
\newcommand {\BY}{{\bf by\ }}
\newcommand {\LOOP}{{\bf loop\ }}
\newcommand {\WHILE}{{\bf while\ }}
\newcommand {\TO}{{\bf to\ }}
\newcommand {\DOWNTO}{{\bf down to\ }}
\newcommand {\FOR}{{\bf for\ }}
\newcommand {\FOREACH}{{\bf for each\ }}
\newcommand {\RETURN}{{\bf return\ }}
\newcommand {\REPEAT}{{\bf repeat\ }}
\newcommand {\UNTIL}{{\bf until\ }}
\newcommand {\LCOM}{$(\ast \;$}
\newcommand {\RCOM}{$\ast)$}
\newcommand {\GOTO}{{\bf goto\ }}

% The following commands are for use INSIDE math mode

\newcommand {\OP}[1]{\mbox{\sc #1}}
\newcommand {\op}[1]{\mbox{\sc #1}}
\newcommand {\mm}[1]{\mbox{\rm #1}\;}
\newcommand {\id}[1]{\mbox{\it #1\ }}

\newcommand {\ASSIGN}{\leftarrow }
\newcommand {\MIN}{\OP{min} }
\newcommand {\MOD}{\; {\bf{\rm mod}} \; } 
\newcommand {\LAMBDA}{{\bf \lambda\ }}
\newcommand {\FALSE}{{\em FALSE\ }}

% Miscellaneous notation

\newcommand {\And}{\wedge}
\newcommand {\Or}{\vee}

\newcommand {\thus}{{\dot{. \: .}\;}}

\newcommand {\bigO}[1]{{\cal O}(#1)}

\newcommand {\app}{\!\!:\!}
\newcommand {\hastype}{::}
\newcommand{\hast}{:}
\newcommand{\qt}[1]{\mbox{``#1''}}
\newcommand{\TexComment}[1]{}

\newcommand{\dq}{\m{\tt "}}
\newcommand{\flatqt}[1]{\m{\dq #1 \dq}}
\newcommand{\seq}{\subseteq}
\newcommand{\derives}{\vdash}

% for writing grammars
\newcommand{\grammar}{::=}
\newcommand{\gor}{\,|\,}

% macros for writing inference rules
\newcommand{\infrule}[2]{\displaystyle{\displaystyle\strut{#1}} \over %
                                        {\displaystyle\strut {#2}}}
\newcommand{\cinfrule}[3]{\parbox{14cm}{\hfil$\infrule{#1}{#2}$\hfil}\parbox{4cm}{$\,#3$\hfil}}


\begin{document}
\handout{4}{3}{Programming Assignment II \\ Due Tuesday, September 26}


%\maketitle

% Three macros for defining:
% 	Unix elements: filenames and program (sans serif)
%	Cool elements: literal tokens (typewriter)
%	C elements: function and variable names (boldface)
%
\def\U#1{{\sf{}#1}}
\def\S#1{{\tt{}#1}} % NB: we often use \verb+...+ for this also
\def\C#1{{\bf{}#1}}


\section{Overview}

Programming assignments II--V will lead you to design and build a compiler
for Cool.  Each assignment will cover one component of the compiler:
lexical analysis, parsing, semantic analysis, and code generation.

For this assignment you are to write a lexical analyzer, also called a
{\em scanner}, using the tool \U{flex}.  You will describe the set of
tokens for Cool in \U{flex} input format.  If you have not done so
already, you should consider buying the reader available for sale from
Copy Central on Euclid Ave.  In addition to having manual pages for
\U{flex} and \U{gmake}, the reader contains the \U{bison} documentation,
which will be used in the next assignment.

You must work in a group for this assignment (where a group consists of
from one to three people), and in order to turn in the assignment,
you must register the group.  To register your group, send a message
to \U{cs164@cory} with the subject \U{PA2 group}, and the message consisting
of the login names of the people in your group, one name per line, and
nothing else.  Each member of the group will receive a confirmation via
email.


\section{Files and Directories}

To get started, you should type
\begin{verbatim}
gmake -f ~cs164/assignments/PA2/Makefile
\end{verbatim}
in a directory where you want to do the assignment.  This command
will copy a number of files to your directory.  Some of the files
will be copied read-only (using symbolic links).  You should not
edit these files.  In fact, if you make and modify private
copies of these files, you may find it impossible to complete the
assignment.  See the instructions in the \U{README} file.

The files that you will need to modify are:
\begin{itemize}
\item \U{cool.flex} \\
This file contains a token start (no pun intended) at a \U{flex} description
for Cool. You can actually build a scanner with this description but
it does not do much. You should read the man pages
for \U{flex} to figure out what this description does do.
Any auxiliary C++ routines that you wish to write should be added directly
to the \U{cool.flex} file after the last \U{\%\%}.

\item \U{test.cl} \\
This file contains some sample input to be scanned. It does not
exercise all of the lexical specification but it is nevertheless an
interesting test.  It is not a good test to start with, nor does it
provide adequate testing of your scanner.  Part of your assignment is
to come up with good testing inputs and testing strategy. 

You should modify this file with tests that you think adequately
exercise your scanner.  Our \U{test.cl} is actually close to a real Cool
program, but your tests need not be.  You may keep as much or as little of our
test as you like.

\item \U{README}\\
This file contains detailed instructions for the assignment.  You should
also edit this file to include the write-up for your
project.  You should explain design decisions, why your code
is correct, and why your test cases are adequate.  It is part of the
assignment to clearly and concisely explain things in text as well as
to comment your code.

\end{itemize}

Although these files are incomplete, the two program files do compile and
run.  There are a number of useful tips on using \U{flex} in the \U{README}
file. 

All of the software supplied with this assignment is supported on both
the HP and DEC machines.  However, if you switch platforms 
be sure to run \U{gmake clean} to remove files compiled for the
other architecture.  

{\bf Important:} You should make sure to place \U{~cs164/bin} at the
beginning of your \U{path} variable to make sure the executables used
are the ones the assignments are designed for.  To do this, add the
line
\begin{verbatim}
setenv PATH ~cs164/bin:${PATH}
\end{verbatim}
at the end of your
\U{.login} file.


\section{Scanner Results}

You should follow the specification of the lexical structure of Cool
given in the Section~10 and Figure 1 of the CoolAid.  Your scanner
should be robust---it should work for any conceivable input.  For
example, you must handle errors such as an EOF occurring in the middle
of a string or comment, as well as string constants that are too long.
These are just some of the errors that can occur; see the manual for
the rest.

You must make some provision for graceful termination if a
fatal error occurs. Core dumps are unacceptable.

Your scanner should maintain the global variable \C{curr\_lineno}
that indicates which line in the source text is currently being scanned.
This feature will aid the parser in printing useful error messages.  

Each call on the scanner returns the next token and lexeme from the
input.  The value returned by the function \C{cool\_yylex} is an integer
code representing the syntactic category: whether it is an integer
literal, semicolon, the \S{if} keyword, etc.  The codes for all tokens
are defined in the file \U{cool-parse.h}.  The second component, the
semantic value or lexeme, is placed in the global union \C{cool\_yylval},
which is of type YYSTYPE.  The type YYSTYPE is also defined in
\U{cool-parse.h}.  The tokens for single character symbols
 (e.g., ``;'' and ``,'', among
others) are represented just by the integer value of the character itself.
All of the single character tokens are listed it the grammar for Cool
in the CoolAid.

Programs tend to have many occurrences of the same lexemes.  For
example, an identifier generally is referred to more than once in
a program (or else it isn't very useful!).  To save space and time, a
common compiler practice is to store lexemes in a {\em string table}.
We provide you with a string table package, which is discussed in detail
in {\em A Tour of the Cool Support Code} and documentation in the code.
For the moment, we only need to know that the type of string table entries
is \C{Symbol}.

For class identifiers, object identifiers, integers and strings,
the semantic value should be a \C{Symbol} stored in the field
\C{cool\_yylval.symbol}.  For boolean constants, the semantic value is stored
in the field \C{cool\_yylval.boolean}.  Except for errors (see below), the 
lexemes for the other tokens do not carry any interesting information.

All errors will be passed along to the parser, which is better
equipped to handle them.  The Cool parser knows about a special error
token called \C{ERROR}.  When an invalid character is encountered,
that character and any invalid characters that follow should be
gathered together into a string until the lexer finds a character that
can begin a new token.  The routine \C{cool\_yylex} should return the token
\C{ERROR}.  The semantic value is the string of illegal characters,
which is stored in the field \C{cool\_yylval.error\_msg} (note that this
field is an ordinary string, not a symbol).  For errors besides
strings of invalid characters (e.g., a string constant that is too
long, or an end-of-file inside of a comment) it is sufficient to set
\C{cool\_yylval.error\_msg} to an informative error message (e.g., ``String
constant too long'' or ``EOF in comment'').  Make sure that the error
message is informative so that we can understand what you did.

There is an issue in deciding how to handle the special identifiers
for the basic classes, {\tt SELF\_TYPE}, and {\tt self}.  However,
this issue doesn't actually come up until later phases of the
compiler---the scanner should treat the special identifiers exactly
like any other identifier.

Finally, if a \U{flex} specification is incomplete (some input has no
regular expression that matches) then the generated scanner will invoke
a default action on unmatched strings.  The default action simply copies
the string to \U{stdout}.  Your final scanner should have no default
actions.  Note that default actions are very bad for \U{mycoolc}, which
works by piping output from one compiler phase to the next; any
extra output will cause errors in downstream phases.

\section{Testing the Scanner}

There are two ways that you can test your scanner.  The first is to
generate sample inputs and run them using \C{lextest} with the
\C{-v} option, which will print out the lexeme and line number of every token 
recognized by your scanner.  When you think your scanner is working,
you should try \U{gmake parser}.  This command will link your scanner
with the course Cool parser.  Once you have successfully linked your
scanner with the parser, {\tt mycoolc} will be a complete Cool compiler
that you can try on the sample programs and your program from Assignment I.

\section{What to Turn in}

When you are ready to turn in the assignment, type \U{gmake turnin} in
the directory where you have prepared your assignment.  This action
will copy the three files of the assignment (\U{cool.flex},
\U{README}, \U{test.cl}) and
\U{test.output} (the output
of running your program on \U{test.cl}) to the reader's
directory.

Doctoring the output that is sent is considered cheating; if you want
to explain something, do it in the \U{README} file.

The last turnin you do will be the one graded.
Each turnin overwrites the previous one.  Remember that there is a
0.5\% penalty per hour for late assignments and that no credit will be
given for an assignment after a solution has been provided.
The burden of convincing us that you understand the material is on you.
Obtuse code, output, and write-ups will have a negative effect on your
grade. Take the extra time to clearly (and concisely!) explain your results.

\end{document}


