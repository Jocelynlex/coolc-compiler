\documentstyle[11pt,handout]{article}
%
% Copyright (c) 1995-1996 by Alex Aiken.  All rights reserved.
% Permission is granted to modify and distribute this document for
% for non-commercial purposes, so long as this copyright notice is retained
% in all copies.
%
% Side margins:
% Actual margin is 1 in + this number
\oddsidemargin -0.25in
\evensidemargin -0.25in

% Text width:
\textwidth 6.9in

% Top margin:
% Actual margin is 1.5 in + this number
\topmargin -.3in

% Text height:
\textheight 8.7in


\newcommand{\attr}[3]{#1:#2\leftarrow#3}
\newcommand{\classmap}[2]{class(#1) = (#2)}
\newcommand{\ossimple}[6]{#1,#2,#3\vdash #4 : #5,#6}
\newcommand{\osrule}[8]{\frac{#7}{\ossimple{#1}{#2}{#3}{#4}{#5}{#6}}\eqno
\mbox{#8}}
\def\U#1{{\sf{}#1}}
\def\S#1{{\tt{}#1}} % NB: we often use \verb+...+ for this also
\def\C#1{{\bf{}#1}}
\newcommand{\dq}{\mbox{\tt "}}

\begin{document}

\handout{12}{8}{Programming Assignment V \\
Due Friday, December 1}

{\bf IMPORTANT}: The late
policy for this assignment is slightly different than for previous
assignments.  As always, you may turn in the assignment any number of
times after the deadline, with the corresponding late penalty.  However, we may grade
whatever you have turned in at any time after the deadline.
The reason for this change is that we need to grade the
last assignment promptly.

\section{Introduction}

In this assignment you will implement code generation for Cool.
This assignment is the end of the line: when completed, you will
have a fully functional Cool compiler.

The code generator makes use of the AST constructed in PA3
and static analysis performed in PA4.  Your code generator should
produce MIPS assembly code that faithfully implements any correct
Cool program.  There is no error recovery in code generation---all
erroneous Cool programs have been detected by the front-end phases
of the compiler.

As with the static analysis assignment, this assignment has much
room for design decisions.  Your program is correct if it generates
correct code; how you achieve that goal is up to you.  We will suggest
certain conventions that we believe make life easier, but you don't
have to take our advice.  As always, explain and justify your design
decisions in the \U{README} file.
This assignment is comparable in size and difficulty to the previous
programming assignment.  Start early!

\section{Files and Directories}
To get the assignment type
\begin{verbatim}
make -f ~cs164/assignments/PA5/Makefile
\end{verbatim}
in a directory named \U{PA5}.  This command copies a number
of files to your directory, some of them with read-only permission.
As usual, you should not modify files that are read-only.
Please read and follow the directions in the \U{README} file.

The files that you may need to modify are:
\begin{itemize}

\item \U{cgen.cc}
This file will contain your code generator.  We have provided an implementation
of some aspects of code generation; studying this code will help you write
the rest of the code generator.  It includes a call to code that will build
an inheritance graph from the provided AST.  You can use the provided code
or replace it with your own.

\item \U{cgen.h}
This file is the header for the code generator.  You may add anything you
like to this file.  It provides classes for implementing the inheritance
graph.  You may replace or modify them as you wish.

\item \U{cgen\_supp.cc} 
This file contains general support code for the code generator.  You will
find a number of handy functions here.  Modify the file as you
see fit, except for the functions \C{emit\_string\_constant},
\C{ascii\_mode}, and \C{emit\_string\_constant}.

\item \U{emit.h}
This file contains code generation macros.  You may modify this file.

\item \U{cool-tree.h} \U{cool-tree.handcode.h} \U{cool-tree.cc}
As usual, these files contain the definitions of classes for AST nodes.
You can add field declarations to the classes in \U{cool-tree.h} or
\U{cool-tree.handcode.h}, but add the definitions of the
methods to \U{cgen.cc}; \U{cool-tree.cc} should not be modified.

\item \U{symtab.h}
The symbol table manager should be adequate as is for code generation, but
you are free to change it.

\item \U{example.cl}
This file should contain a test program of your own design.  
Test as many features of the code generator as you can manage to 
fit into one file.

\item \U{README}
This file will contain the write-up for your assignment. 
It is critical that you explain design decisions, how your
code is structured, and why you believe your design is a good one
(i.e., why it leads to a correct and robust program).  It is part of
the assignment to explain things in text as well as to comment your
code. 

\end{itemize}

There will be many other files in your directory containing other
pieces of the compiler.  These files were described in previous
handouts and are listed in the \U{README} file.  The files we will
collect and use to grade your assignment are the ones listed above;
you should not modify any other files.

\section{Designing and Testing the Code Generator}

You will need a working lexer, parser, and semantic analyzer to test
your code generator.  See the \U{README} file for instructions on how
to use either your own components or the components from \U{coolc}.
It is wise to test your code generator with the \U{coolc} lexer,
parser, and semantic analyzer at least once, because we will grade
your code generator using \U{coolc}'s version. 

See the \U{README} file for instructions for compiling and running
a compiler with your code generator.  For your convenience, a command
line debugging flag \C{-c} is included in the skeleton, which controls
the global variable \C{cgen\_debug}.

There are many possible ways to write the code generator.  One
reasonable strategy is to perform code generation in two passes.  The
first pass decides the object layout for each class, particularly the
offset at which each attribute is stored in an object.  Using this
information, the second pass recursively walks each feature and
generates stack machine code for each expression.

There are a number of things you must keep in mind while designing
your code generator. First, your code generator must work correctly
with the Cool runtime system, which is explained in the {\em Cool Tour}
handout.  Second, you should have a clear picture of
the runtime semantics of Cool programs.  The semantics are described
informally in the first part of the {\em CoolAid}, and a precise
description of how Cool programs should behave is given in Section 13
of the manual.  Third, you
should understand the MIPS instruction set.  An overview of MIPS
operations is given in the \U{spim} documentation, which is in the
course reader and on the class Web page.
Fourth, you should decide what invariants your generated
code will observe and expect; i.e., what registers will be saved,
which might be overwritten, etc.  You may also find it useful to refer
to information on code generation in the lecture notes and the text.

\section{Extra Credit}

There are two ways to earn extra credit on this assignment.  The first is
the ordinary way: find and document a bug in \U{coolc}.  The second way
is to implement some optimization in your compiler.

Extra credit will be awarded for projects that, in addition to code
generation, perform some significant optimization of the code.  The
amount of extra credit depends on how well the optimization is
written, documented, and demonstrated.  Two critical factors are: (1)
correctness (the optimizations don't result in incorrect programs) and
(2) the percentage speedup your optimized code achieves over
\U{coolc}, as measured in the number of instructions executed on
\U{spim} over a suite of benchmarks of our choosing.

The total extra credit for optimization will not exceed 5\% of the 
total grade for the course.  Roughly speaking, the extra credit is
worth up to about half of the two large programming assignments.
The final curve for the course will be determined {\em before} including
the extra credit.  In other words, if you elect not to do an optimization
phase, you will not be at a disadvantage in the final grading with respect
to those who do.

This extra-credit option is open-ended; you can do as much as you
like.  We will award credit for results. A project that merely attempts, 
but does not complete, an optimization phase may receive as little
as no extra credit.

There are many possible optimizations to implement; see the text for
ideas.  Assuming your initial code generator is straightforward (like
coolc's), then two directions that may yield significant
improvement are (1) improving register
usage and
(2) eliminating the test for void in dynamic dispatches if it is safe
to do so.

{\bf WARNING.} We have not implemented an optimization phase in \U{coolc},
so we can't be sure how difficult it will be.  For the same reason, we
have no skeleton code to give you---you are on your own.  If you want to
do an optimization phase, you are encouraged to talk it over with one
of the course staff first. {\em Under absolutely no circumstances should you try
optimization before your code generator is finished!!}

There is a {\tt -O} flag that controls the global variable
 \C{cgen\_optimize}.  If you do an optimization phase, it should have
 no effect unless \C{cgen\_optimize} is 1.  We
 will grade your code generator first with optimization off; this will
 prevent you from losing points due to bugs in your optimizer.

\section{Garbage Collection}

To receive full credit for this assignment, your code generator must work
correctly with the generational garbage collector in the Cool runtime
system.  The skeleton \U{cgen.cc}  contains a function
\U{code\_select\_gc} that generates code that sets GC options
from command line flags.
The command line flags that affect garbage collection are \U{-g},
\U{-t}, and \U{-T}.  Garbage collection is disabled by default; the
flag \U{-g} enables it.  When enabled, the garbage collector
not only reclaims memory, but also verifies that ``-1'' separates all objects in the heap, thus checking
that the program (or the collector!) has not accidentally overwritten
the end of an object.  The \U{-t} and \U{-T} flags are used for additional
testing.  With \U{-t} the collector
performs collections very frequently (on every allocation).  The garbage
does not directly use \U{-T}; in \U{coolc} the \U{-T} option causes extra
code to be generated that performs more runtime validity checks.  You
are free to use (or not use) \U{-T} for whatever you wish.

For your implementation, the simplest way to start is not to use the
collector at all (this is the default).   When you decide to use the
collector, be sure to carefully review the garbage collection interface
described in the {\em Cool Tour}.   Ensuring that your code generator
correctly works with the garbage collector in {\em all} circumstances
is not trivial.



\section{Spim and XSpim}
You will find \U{spim} and \U{xspim} useful for debugging your generated code.
\U{xspim} works like \U{spim} in that it lets you run MIPS
assembly programs. However, it has many features that allow you to
look at the memory, registers, data segment, and code segment of the
code. You can also set breakpoints and single step your program.
Look at the documentation for \U{spim}/\U{xspim} in the course reader
or in the cs164 home page.

{\bf Warning}: One thing that makes debugging with \U{spim} difficult
is that \U{spim} is an interpreter for assembly code and not a true assembler.
If your code or data definitions refer to undefined labels,
the error shows up only if the executing code actually refers to
such a label. Moreover, an error is reported only for undefined
labels that appear in the code section of your program. If you have
constant data definitions that refer to undefined labels, \U{spim} won't
tell you anything. It will just assume the value 0 for such undefined
labels.


\end{document}
